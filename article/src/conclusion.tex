\subsection{Synthèse et perspectives}

L'approche par triplet de probabilités représente une évolution significative par rapport à l'approche par pentes fixes, offrant une adaptabilité et une flexibilité accrues. Son principal atout réside dans sa capacité à produire des comportements plus nuancés tout en maintenant une dominance de l'action de "maintien" qui assure la stabilité thermique.

\subsubsection{Perspectives d'amélioration}

Plusieurs pistes d'amélioration se dégagent de ces travaux:

\paragraph{Améliorations communes aux deux approches}
\begin{itemize}
    \item \textbf{Initialisation intelligente}: Démarrer avec des distributions de probabilités informées par des connaissances préalables
    \item \textbf{Intégration d'autres signaux}: Combiner plusieurs types de capteurs pour enrichir le modèle d'apprentissage
    \item \textbf{Validation avec utilisateurs réels}: Déployer ces approches dans des environnements réels pour valider leur efficacité
\end{itemize}

\paragraph{Perspectives spécifiques à l'approche par triplet}
\begin{itemize}
    \item \textbf{Taille de pas adaptative}: Faire varier la magnitude des actions de diminution/augmentation en fonction du contexte
    \item \textbf{Métaparamètres dynamiques}: Ajuster automatiquement $\gamma$ en fonction des préférences observées de l'utilisateur
    \item \textbf{Approche multi-agents}: Combiner plusieurs agents LRI spécialisés dans différentes situations d'usage
\end{itemize}

\subsubsection{Extension à d'autres domaines}

Les deux approches, et particulièrement le modèle par triplet de probabilités, pourraient être étendues à d'autres domaines du bâtiment intelligent:

\begin{itemize}
    \item \textbf{Systèmes HVAC complexes}: Gestion adaptative du chauffage/climatisation avec prise en compte de l'inertie thermique
    \item \textbf{Gestion de l'eau chaude}: Optimisation de la température et de la disponibilité de l'eau chaude
    \item \textbf{Contrôle d'accès et sécurité}: Adaptation des systèmes de sécurité aux habitudes des utilisateurs
\end{itemize}

En conclusion, bien que les deux approches démontrent leur efficacité, l'approche par triplet de probabilités semble offrir un paradigme plus prometteur pour les futures applications de bâtiments intelligents, combinant efficacité énergétique, confort utilisateur et adaptabilité aux conditions changeantes. Son caractère stochastique intrinsèque lui confère un avantage notable dans les environnements complexes où le comportement utilisateur présente une forte variabilité et une non-stationnarité.
